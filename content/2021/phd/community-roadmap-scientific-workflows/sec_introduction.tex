\section{Introduction}

Scientific workflows are used almost universally across scientific domains for solving complex and large-scale computing and data analysis problems. The importance of workflows is highlighted by the fact that they have underpinned some of the most significant discoveries of the past decades~\cite{badia2017workflows}. Many of these workflows have significant computational, storage, and communication demands, and thus must execute on a range of large-scale platforms, from local clusters to public clouds and upcoming exascale HPC platforms~\cite{ferreiradasilva-fgcs-2017}. Managing these executions is often a significant undertaking, requiring a sophisticated and versatile software infrastructure. 

Historically, infrastructures for workflow execution consisted of complex, integrated systems, developed in-house by workflow practitioners with strong dependencies on a range of legacy technologies---even including sets of ad-hoc scripts. Due to the increasing need to support workflows, dedicated workflow systems were developed to provide abstractions for creating, executing, and adapting workflows conveniently and efficiently while ensuring portability.  While these efforts are all worthwhile individually, there are now hundreds of independent workflow systems~\cite{workflow-systems}. These workflow systems are created and used by thousands of researchers and developers, leading to a rapidly growing corpus of workflows research publications. The resulting workflow system technology landscape is fragmented, which may present significant barriers for future workflow users due to many seemingly comparable, yet usually mutually incompatible, systems that exist. 

\begin{table*}[!th]
\centering
\scriptsize
\caption{Summary of current workflows research and development challenges and proposed community activities.}
\vspace{-8pt}
\begin{tabular}{p{1.3cm}p{7.7cm}p{7.7cm}}
\toprule 
Theme & Challenges & Community Activities \\
\midrule
\makecell[l]{FAIR\\Computational\\Workflows} &
\bp{7.7cm}{
    FAIR principles for computational workflows that consider the complex lifecycle from specification to execution and data products
    \\
    Metrics to measure the ``FAIRness'' of a workflow
    \\
    %Engage the community to define 
    Principles, policies, and best practices
}
&
\bp{7.7cm}{
    Review prior and current efforts for FAIR data and software with respect to workflows, and outline principles for FAIR workflows
    \\
    Define recommendations for FAIR workflow developers and systems
    \\
    Automate FAIRness in workflows by recording necessary provenance data
}
\\
\midrule

\makecell[l]{AI\\Workflows} &
\bp{7.7cm}{
    Support for heterogeneous compute resources and fine-grained data management features, versioning, and data provenance capabilities
    \\
    Capabilities for enabling workflow steering and dynamic workflows
    \\
    Integration of ML frameworks into the current HPC landscape
}
&
\bp{7.7cm}{
    Develop comprehensive use cases for sample problems with representative workflow structures and data types
    \\
    Define a process for characterizing the challenges for enabling AI workflows
    \\
    Develop AI workflows as a way to benchmark HPC systems
}
\\
\midrule

\makecell[l]{Exascale\\Challenges\\and Beyond} &
\bp{7.7cm}{
    Resource allocation policies and schedulers are not designed for workflow-aware abstractions, thus users tend to use an ill-fitted job abstraction
    \\
    Unfavorable design of resource descriptions and mechanisms for workflow users/systems, and lack of fault-tolerance and fault-recovery solutions
}
&
\bp{7.7cm}{
    Develop documentation in the form of workflow templates/recipes/miniapps for execution on high-end HPC systems
    \\
    Specify benchmark workflows for exascale execution
    \\
    Include workflow requirements as part of the machine procurement process
}
\\
\midrule

\makecell[l]{APIs, Reuse,\\Interoperability,\\and Standards} &
\bp{7.7cm}{
    Workflow systems differ by design, thus interoperability at some layers is likely to be more impactful than others
    \\
    Workflow standards are typically developed by a subset of the community
    \\
    Quantifying the value of common representations of workflows is not trivial
}
&
\bp{7.7cm}{
    Identify differences and commonalities between different systems
    \\
    Identify and characterize domain-specific efforts, identify workflow patterns, and develop case-studies of business process workflows and serverless workflow systems
}
\\
\midrule

\makecell[l]{Training\\and Education} &
\bp{7.7cm}{
    Many workflow systems have high barriers to entry and lack training materials
    \\
    Homegrown workflow solutions and constraints can prevent users from reproducing their functionality on workflow systems developed by others
    \\
    Unawareness of the workflow technological and conceptual landscape
}
&
\bp{7.7cm}{
    Identify basic sample workflow patterns, develop a community workflow knowledge-base, and look at current research on technology adoption
    \\
    Include workflow terminology and concepts in university curricula and software carpentry efforts
}
\\
\midrule

\makecell[l]{Building\\a Workflows\\Community} &
\bp{7.7cm}{
    Diverse definitions of a ``workflows community''
    \\
    Remedy the inability to link developers and users to bridge translational gaps
    \\
    Pathways for participation in a network of researchers, developers, and users
}
& 
\bp{7.7cm}{
    Establish a common knowledge-base for workflow technology
    \\
    Establish a \emph{Workflow Guild}: an organization focused on interaction and relationships, providing self-support between workflow developers and their systems
}
\\
\bottomrule
\end{tabular}
\label{tab:challenges}
\vspace{-10pt}
\end{table*}

In the current workflow research, there are conflicting theoretical bases and abstractions for what constitutes a workflow system. It may be possible to translate between systems that use the same underlying abstractions; however, the contrary is not feasible. Specifically, typical systems have a layered model that abstractly underlie them: (i)~if the models are the same for two systems, they are compatible to some extent, and if they implement the same layers, they can be interchanged (modulo some translation effort); (ii)~if the models are the same for two systems, but they are implemented by components at different layers, they can be complementary, and may have common elements that could be shared; (iii)~if the models are distinct, workflows or system components are likely not exchangeable or interoperable. As a result, many teams still elect to build their own custom solutions rather than adopt, adapt, or build upon, existing workflow systems. This current state of the workflow systems landscape negatively impacts workflow users, developers, and researchers~\cite{deelman2018future}.

The WorkflowsRI~\cite{workflowsri} and ExaWorks~\cite{al2021exaworks} projects have partnered to bring the workflows community (researchers, developers, science and engineering users, and cyberinfrastructure experts) together to collaboratively elucidate the R\&D efforts necessary to remedy the above situation. They conducted a series of virtual events entitled ``Workflows Community Summits,'' in which the overarching goal was to (i)~develop a view of the state of the art, (ii)~identify key research challenges, (iii)~articulate a vision for potential activities, and (iv)~explore technical approaches for realizing (part of) this vision. The summits gathered over 70 participants, including lead researchers and developers from around the world, and spanning distinct workflow systems and user communities. 
The outcomes of the summits have been compiled and published in two technical reports~\cite{ferreiradasilva2021wcs, wcs2021technical}. In this paper, we summarize the discussions and findings by presenting a consolidated view of the state of the art, challenges, and  potential efforts, which we synthesize into a community roadmap. Table~\ref{tab:challenges} presents, in the form of top-level themes, a summary of those challenges and targeted community activities. Table~\ref{tab:roadmap} summarizes a proposed community roadmap with technical approaches. 

The remainder of this paper is organized as follows. Sections~\ref{sec:fair}-\ref{sec:community} provide a brief state of the art and challenges for each theme and proposed community activities. Section~\ref{sec:roadmap} discusses technical approaches for a community roadmap. Section~\ref{sec:conclusion} concludes with a summary of  discussions.

\begin{table*}[!th]
\centering
\scriptsize
\caption{Summary of technical roadmap milestones per research and development thrust.}
\vspace{-8pt}
\begin{tabular}{p{2.8cm}p{14.4cm}}
\toprule 
Thrust & Roadmap Milestones \\
\midrule
\makecell[l]{Definition of common\\workflow patterns and\\benchmarks} &
\bp{14.4cm}{
    Define small sets of workflow patterns and benchmark deliverables, and implement them using a selected set of workflow systems
    \\
    Investigate automatic generation of patterns and configurable benchmarks (e.g., to enable weak and strong scaling experiments)
    \\
    Establish or leverage a centralized repository to host and curate patterns and benchmarks
}
\\
\midrule

\makecell[l]{Identifying paths toward\\interoperability of workflow\\systems} &
\bp{14.4cm}{
    Define interoperability for different roles, develop a horizontal interoperability (i.e., making interoperable components), and establish a requirements document per abstraction layer
    \\
    Develop real-world workflow benchmarks, use cases for interoperability, and common APIs that represent workflow library components
    \\
    Establish a workflow systems developer community
}
\\
\midrule

\makecell[l]{Improving workflow systems'\\interface with legacy and\\ emerging HPC software and\\hardware stacks} &
\bp{14.4cm}{
    Document a machine-readable description of key properties of widely used sites, and remote authentication needs from the workflow perspective
    \\
    Identify new workflow patterns (e.g., motivated by AI workflows), attain portability across heterogeneous hardware, and develop a registry of execution environment information
    \\
    Organize a community event involving workflow system developers, end users, authentication technology providers, and facility operators
}
\\
\bottomrule
\end{tabular}
\label{tab:roadmap}
\vspace{-10pt}
\end{table*}

