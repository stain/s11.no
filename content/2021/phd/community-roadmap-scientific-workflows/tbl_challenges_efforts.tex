\begin{table*}[!th]
\centering
\scriptsize
\caption{Summary of current workflows research and development challenges and proposed community activities.}
\vspace{-8pt}
\begin{tabular}{p{1.3cm}p{7.7cm}p{7.7cm}}
\toprule 
Theme & Challenges & Community Activities \\
\midrule
\makecell[l]{FAIR\\Computational\\Workflows} &
\bp{7.7cm}{
    FAIR principles for computational workflows that consider the complex lifecycle from specification to execution and data products
    \\
    Metrics to measure the ``FAIRness'' of a workflow
    \\
    %Engage the community to define 
    Principles, policies, and best practices
}
&
\bp{7.7cm}{
    Review prior and current efforts for FAIR data and software with respect to workflows, and outline principles for FAIR workflows
    \\
    Define recommendations for FAIR workflow developers and systems
    \\
    Automate FAIRness in workflows by recording necessary provenance data
}
\\
\midrule

\makecell[l]{AI\\Workflows} &
\bp{7.7cm}{
    Support for heterogeneous compute resources and fine-grained data management features, versioning, and data provenance capabilities
    \\
    Capabilities for enabling workflow steering and dynamic workflows
    \\
    Integration of ML frameworks into the current HPC landscape
}
&
\bp{7.7cm}{
    Develop comprehensive use cases for sample problems with representative workflow structures and data types
    \\
    Define a process for characterizing the challenges for enabling AI workflows
    \\
    Develop AI workflows as a way to benchmark HPC systems
}
\\
\midrule

\makecell[l]{Exascale\\Challenges\\and Beyond} &
\bp{7.7cm}{
    Resource allocation policies and schedulers are not designed for workflow-aware abstractions, thus users tend to use an ill-fitted job abstraction
    \\
    Unfavorable design of resource descriptions and mechanisms for workflow users/systems, and lack of fault-tolerance and fault-recovery solutions
}
&
\bp{7.7cm}{
    Develop documentation in the form of workflow templates/recipes/miniapps for execution on high-end HPC systems
    \\
    Specify benchmark workflows for exascale execution
    \\
    Include workflow requirements as part of the machine procurement process
}
\\
\midrule

\makecell[l]{APIs, Reuse,\\Interoperability,\\and Standards} &
\bp{7.7cm}{
    Workflow systems differ by design, thus interoperability at some layers is likely to be more impactful than others
    \\
    Workflow standards are typically developed by a subset of the community
    \\
    Quantifying the value of common representations of workflows is not trivial
}
&
\bp{7.7cm}{
    Identify differences and commonalities between different systems
    \\
    Identify and characterize domain-specific efforts, identify workflow patterns, and develop case-studies of business process workflows and serverless workflow systems
}
\\
\midrule

\makecell[l]{Training\\and Education} &
\bp{7.7cm}{
    Many workflow systems have high barriers to entry and lack training materials
    \\
    Homegrown workflow solutions and constraints can prevent users from reproducing their functionality on workflow systems developed by others
    \\
    Unawareness of the workflow technological and conceptual landscape
}
&
\bp{7.7cm}{
    Identify basic sample workflow patterns, develop a community workflow knowledge-base, and look at current research on technology adoption
    \\
    Include workflow terminology and concepts in university curricula and software carpentry efforts
}
\\
\midrule

\makecell[l]{Building\\a Workflows\\Community} &
\bp{7.7cm}{
    Diverse definitions of a ``workflows community''
    \\
    Remedy the inability to link developers and users to bridge translational gaps
    \\
    Pathways for participation in a network of researchers, developers, and users
}
& 
\bp{7.7cm}{
    Establish a common knowledge-base for workflow technology
    \\
    Establish a \emph{Workflow Guild}: an organization focused on interaction and relationships, providing self-support between workflow developers and their systems
}
\\
\bottomrule
\end{tabular}
\label{tab:challenges}
\vspace{-10pt}
\end{table*}