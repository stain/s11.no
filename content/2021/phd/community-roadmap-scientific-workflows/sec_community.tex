\section{Building a Workflows Community}
\label{sec:community}

Given the current large size and fragmentation of the workflow technology landscape, there is a clear need to establish a cohesive community of workflow developers and users. This community would be crucial for avoiding unnecessary duplication of effort and would allow for sharing, and thus growing, of knowledge. To this end, there are four main components that need to be addressed for building a community: (i)~identity building, (ii)~trust, (iii)~participation, and (iv)~rewards.


%% Subsection
\subsection{Brief State-of-the-art and Challenges}

The most natural idea is to think of two \textbf{\emph{distinct communities}}: (i)~a Workflow Research and Development Community, and (ii)~a Workflow User Community. The former gathers people who share interest in workflow research and development, and corresponding sub-disciplines. Subgroups of this community are based on common methodologies, technical domains (e.g., computing, provenance, and design), scientific disciplines, as well as geographical and funding areas. The latter gathers anyone using workflows for optimization of their work processes. However, most domain scientists prioritize rigor on their domain-specific objectives, often viewing workflow development as a means to a solution.
% However, most domain science users think of themselves in their specific disciplines first, as they just happen to use workflows to get their work done.

The two aforementioned communities are not necessarily disjoint, but currently have little overlap. And yet, it is crucial that they interact. Such interaction seems to happen only on a case-by-case basis, rather than via organized community efforts.  One could, instead, envision a single community (e.g., team of users, or ``team-flow'') that gathers both workflow system developers and workflow-focused users, with the common goal of spreading knowledge and adoption of workflows, thus working towards increased \textbf{\emph{sharing and convergence/interoperation}} of technologies and approaches.

Establishing trust and processes is key for bringing both communities together. There is no one-size-fits-all workflow system or solution for all domains, instead each domain presents their own specific needs and have different preferred ways to address problems. There is a pressing need for \textbf{\emph{maintaining documentation and dissemination}} that fits different usage options and needs.


%% Subsection
\subsection{A Vision for Potential Community Activities}

Given the above, there are several existing community efforts that could serve as inspiration, for example, the WorkflowHub Club~\cite{workflowhub} and Galaxy~\cite{galaxy}. One approach is to gather experience from computing facilities where teams have successfully adopted and are successfully running workflow systems~\cite{nersc-workflows}. Another possibility is to use proposal/project reviews as mechanisms for spreading workflow technology knowledge. Specifically, finding ways to make proposal authors (typically domain scientists) aware of available technology would prevent their proposed work from ``re-inventing the wheel.'' Finally, it is clear that solving the ``community challenge'' has large overlap with solving the ``education challenge'' (Section~\ref{sec:training}).

A short-term activity would entail \textbf{\emph{establishing a common knowledge-base for workflow technology}} so that workflow users could navigate the current technology landscape. User criteria (for navigation) need to be defined. Workflow system developers can add to this knowledge base via self-reporting and could include test statuses for a set of standard workflow configurations, especially if workflow systems are deployed across sites. There is large overlap with similar proposed community efforts identified in Sections~\ref{sec:exascale} and~\ref{sec:training}.

An ambitious vision would be to \textbf{\emph{establish a ``Workflow Guild''}}, an organization focused on interactions, relationships, and self-support between subscribing workflow developers and their systems, as well as dissemination of best-practices and tools that are used in the development and use of these systems. However, there are still barriers to be conquered: (i)~such a community could be too self-reflecting, and yet still remain fragmented; (ii)~a cultural/social problem is that creating a new system is typically more exciting for computer scientists as opposed to re-using an existing system; and (iii)~building trust and reducing internal competition will be difficult, though building community identity will help the Guild work together against external competitors.
